% This is "sig-alternate.tex" V2.0 May 2012
% This file should be compiled with V2.5 of "sig-alternate.cls" May 2012
%
% This example file demonstrates the use of the 'sig-alternate.cls'
% V2.5 LaTeX2e document class file. It is for those submitting
% articles to ACM Conference Proceedings WHO DO NOT WISH TO
% STRICTLY ADHERE TO THE SIGS (PUBS-BOARD-ENDORSED) STYLE.
% The 'sig-alternate.cls' file will produce a similar-looking,
% albeit, 'tighter' paper resulting in, invariably, fewer pages.
%
% ----------------------------------------------------------------------------------------------------------------
% This .tex file (and associated .cls V2.5) produces:
%       1) The Permission Statement
%       2) The Conference (location) Info information
%       3) The Copyright Line with ACM data
%       4) NO page numbers
%
% as against the acm_proc_article-sp.cls file which
% DOES NOT produce 1) thru' 3) above.
%
% Using 'sig-alternate.cls' you have control, however, from within
% the source .tex file, over both the CopyrightYear
% (defaulted to 200X) and the ACM Copyright Data
% (defaulted to X-XXXXX-XX-X/XX/XX).
% e.g.
% \CopyrightYear{2007} will cause 2007 to appear in the copyright line.
% \crdata{0-12345-67-8/90/12} will cause 0-12345-67-8/90/12 to appear in the copyright line.
%
% ---------------------------------------------------------------------------------------------------------------
% This .tex source is an example which *does* use
% the .bib file (from which the .bbl file % is produced).
% REMEMBER HOWEVER: After having produced the .bbl file,
% and prior to final submission, you *NEED* to 'insert'
% your .bbl file into your source .tex file so as to provide
% ONE 'self-contained' source file.
%
% ================= IF YOU HAVE QUESTIONS =======================
% Questions regarding the SIGS styles, SIGS policies and
% procedures, Conferences etc. should be sent to
% Adrienne Griscti (griscti@acm.org)
%
% Technical questions _only_ to
% Gerald Murray (murray@hq.acm.org)
% ===============================================================
%
% For tracking purposes - this is V2.0 - May 2012

\documentclass{sig-alternate}

\begin{document}
%
% --- Author Metadata here ---
\conferenceinfo{WOODSTOCK}{'97 El Paso, Texas USA}
%\CopyrightYear{2007} % Allows default copyright year (20XX) to be over-ridden - IF NEED BE.
%\crdata{0-12345-67-8/90/01}  % Allows default copyright data (0-89791-88-6/97/05) to be over-ridden - IF NEED BE.
% --- End of Author Metadata ---

\title{XOS : Experimental Operating System }

%
% You need the command \numberofauthors to handle the 'placement
% and alignment' of the authors beneath the title.
%
% For aesthetic reasons, we recommend 'three authors at a time'
% i.e. three 'name/affiliation blocks' be placed beneath the title.
%
% NOTE: You are NOT restricted in how many 'rows' of
% "name/affiliations" may appear. We just ask that you restrict
% the number of 'columns' to three.
%
% Because of the available 'opening page real-estate'
% we ask you to refrain from putting more than six authors
% (two rows with three columns) beneath the article title.
% More than six makes the first-page appear very cluttered indeed.
%
% Use the \alignauthor commands to handle the names
% and affiliations for an 'aesthetic maximum' of six authors.
% Add names, affiliations, addresses for
% the seventh etc. author(s) as the argument for the
% \additionalauthors command.
% These 'additional authors' will be output/set for you
% without further effort on your part as the last section in
% the body of your article BEFORE References or any Appendices.

\numberofauthors{4} %  in this sample file, there are a *total*
% of EIGHT authors. SIX appear on the 'first-page' (for formatting
% reasons) and the remaining two appear in the \additionalauthors section.
%
\author{
% You can go ahead and credit any number of authors here,
% e.g. one 'row of three' or two rows (consisting of one row of three
% and a second row of one, two or three).
%
% The command \alignauthor (no curly braces needed) should
% precede each author name, affiliation/snail-mail address and
% e-mail address. Additionally, tag each line of
% affiliation/address with \affaddr, and tag the
% e-mail address with \email.
%
\\
% 1st author
\alignauthor
Shamil C. M.\\
       \affaddr{National Institue of Technology	}\\
       \affaddr{Calicut, India}\\
       \email{shamil\_bcs09@nitc.ac.in}
% 2nd. author
\alignauthor Sreeraj S\\
       \affaddr{National Institue of Technology	}\\
       \affaddr{Calicut, India}\\
       \email{sreeraj\_bcs09@nitc.ac.in}
\\
% 3rd. author  
\alignauthor 
Vivek Anand T. Kallampally\\
       \affaddr{National Institue of Technology	}\\
       \affaddr{Calicut, India}\\
       \email{vivekanand\_bcs09@nitc.ac.in}
 % use '\and' if you need 'another row' of author names
\\
\and 
% 4th. author
\alignauthor
K. Murali Krishnan\\
       \affaddr{National Institue of Technology	}\\
       \affaddr{Calicut, India}\\
       \email{kmurali@nitc.ac.in}
\\
}
% There's nothing stopping you putting the seventh, eighth, etc.
% author on the opening page (as the 'third row') but we ask,
% for aesthetic reasons that you place these 'additional authors'
% in the \additional authors block, viz.
\additionalauthors{Additional authors: John Smith (The Th{\o}rv{\"a}ld Group,
email: {\texttt{jsmith@affiliation.org}}) and Julius P.~Kumquat
(The Kumquat Consortium, email: {\texttt{jpkumquat@consortium.net}}).}
\date{30 July 1999}
% Just remember to make sure that the TOTAL number of authors
% is the number that will appear on the first page PLUS the
% number that will appear in the \additionalauthors section.

\maketitle
\begin{abstract}
In this paper, we introduce you to an operating system project that helps undergraduate computer science students learn and understand the basics of building an operating system. The specification of XOS or Experimental Operating System has been laid out for students to build it from scratch in a bottom-up manner. XOS runs on a simulated machine hardware with a simplified innate instruction set  and utilizes its own filesystem. Unlike other common instructional operating systems, the complete development environment including custom programming languages, debugger, file system interface and a highly instructive roadmap for sequentially building XOS is provided. A student building XOS will implement features like multiprogramming, file systems, process management and virtual memory management.

\end{abstract}

% A category with the (minimum) three required fields
\category{K.3.2}{Computers and Education}{Computer and Information Science Education}


\terms{Design, Experimentation}

\keywords{XOS, instructional operating system, experimental operating system}

\section{Introduction}

Teaching operating systems has been a challenge at undergraduate level. To tackle this problem several instructional operating systems like Nachos\cite{nachos}, OS/161\cite{os161}, Pintos\cite{Pintos}, GeekOS\cite{survey} etc. have been developed by various universities. Nachos\cite{nachos} has been one of the most popular instructional operating systems available and is being used in many institutes across the world \cite{survey}. Although Nachos implementation is fairly simple, it uses a mixed mode approach, where the operating system kernel is co-resident with the machine simulator and fused together as a single program. This does not provide an intuitive idea of the separation between these components. Moreover Nachos\cite{nachos} and OS/161\cite{os161} runs on top of MIPS machine simulator. For these systems, a user's machine running on other platforms require cross compilers to MIPS. However there are few MIPS tools available for newer machine architectures, and hence there is a need for custom tools. Developing custom tools for MIPS is tedious. The authors of OS/161 has expressed plans of moving towards a different architecture due to lack of freely available tools \cite{os161}. XOS or Experimental Operating System, which we propose as a project for undergraduate operating systems laboratory courses, addresses these issues, by providing an original mahcine architecture known as XSM (Experimental String Machine) which has its own instruction set. This completely new and easy-to-understand instruction set helps avoid complexity associated with actual architecture and helps students to focus on the operating system aspects. The complete package including the high level languages for application as well as system programs, their cross-compilers to XSM machine architecture, simulator and debugger for XSM is provided for building XOS from scratch. In XOS, there is a clear differentiation between the machine and the operating system kernel like OS/161\cite{os161}, which is more realistic towards the actual scenario and has its own set of tools.\\

 
Instructional operating systems like Minix and Xinu \cite{survey}, provide a functional operating system on which modifications are to be done by students. Almost every other instructional operating system provides a skeleton of an operating system. However in XOS, only the specification has been laid out, and students learn to implement XOS from ground up using the tools provided. Most instructional operating systems use C/C++ or Java for programming. Knowledge of a particular high level language becomes necessary for programming the operating system. In this project, a  simple high level language called APL (Application Programmer's Language) and its cross-compiler to XSM instruction set is provided to write user programs to test XOS, and an XSM machine dependant language called SPL(System Programmers Language) and its cross-compiler is provided to program the OS itself. Most instructional operating systems use the UNIX filesystem for file management by the operating system. XOS is different from other instructional operating systems by providing its natve file system known as XFS (Experimental File System). An interface to between the UNIX filesystem and XFS filesystem is also provided.  \\

XOS has features like multiprogramming, process management, filesystem and virtual memory. A sequence of stages are provided in an instructive roadmap which helps students to build XOS sequentially. Although certain simplifications have been made in XOS, compared to real systems like absence of blocking system calls, device management and file caching, the elementary and fundamental aspects of operating systems and data structures have been retained. Process synchronization have been completely left out because, the it will turn out to be too much overwhelming for a student in a 16 week semester. The further sections describe in detail the various components of XOS.


\section{System Components}
As explained in the previous section, an the primary components of the project include a simulated machine hardware (XSM), file system (XFS) and the operating system itself (XFS). Apart from the primary components, various tools have been provided as part of the development environement. They include languages like APL and SPL and their cross compilers to XSM instruction set is also provided, XSM debugger, and a UNIX-XFS interface to transfer files between a UNIX machine and XFS disk. The XFS disk is basically a file in UNIX machine which acts as a disk for XOS. 



\subsection{Experimental String Machine (XSM)}
XOS runs on a simulated machine hardware called XSM or Experimental String Machine. XSM uses an easy-to-understand native 2-address instruction set. The various components of the machine include registers, memory, timer and the disk. A UNIX file simulates the disk for XSM.\\

XSM has a timer which triggers after fixed number of instructions as compared to a timer interval in real machines. An instruction triggered timer was preferred over a clock-triggered timer to ensure that the timer interrupt is not invoked in between the simulation of a single instruction. Thus, an instruction in XSM is always atomic. Instructions are executed one after the other in a non-pipelined manner.\\

XSM has a memory of 64 pages. Size of each page is 512 words. A word is the smallest addressable unit in XSM as compared to a byte in MIPS. XSM is a string machine, and each word is stored internally as strings of size 16 characters.  However, the XSM supports two data types, integer and strings and has instructions for both  the data types. There are two privilege modes in XSM, the user mode and kernel mode. Switching between modes is done by instructions.

\subsection{Experimental File System (XFS)}
The disk for XSM (which is a UNIX file) can be formatted using XFS or Experimental File System. XFS is the file system compatible with XOS. Since file system management is done by students building XOS, the disk organization must be easily understood and at the same time must give an insight on the data structures used in real file systems. Hence we chose to have a native file system for XOS.\\

The disk is formatted with this file system using the interface provided as part of the development environment. XFS is a simple file system with no directory structure. The data is organized into blocks of size equal to the page size in XSM memory. There are 512 blocks in XFS which holds the file and disk data structures, OS routines, user programs and data files. The various data structures in XFS include the Disk Free List which maintains information about used and unused blocks on the disk and the FAT or the File Allocation Table stores details of the files in the disk.

\subsection{Experimental Operating System (XOS)}
The specification for an Experimental Operating System or XOS is laid out to the students. In this project, students will build XOS to meet the specification. XOS is a simulated operating system which runs on top of XSM which is a simulated machine hardware. The OS kernel unlike Nachos \cite{nachos} resides in the memory of the machine during runtime, and is not fused together with the simulated machine and compiled as a single program. The simulated disk formatted using XFS, permanently stores the OS routines and data structures like in real systems. The disk is simulated using a UNIX file. This system is close to real systems and the abstractions are easy to understand. \\

The various components of the operating system include routines like OS startup code, eight interrupt routines including the timer interrupt and the  exception handler routine, process data structures like ready list of PCBs, per-process page tables, the system wide-open file 

\subsection{Development Tools}
Because tables cannot be split across pages, the best
placement for them is typically the top of the page
nearest their initial cite.  To
ensure this proper ``floating'' placement of tables, use the
environment \textbf{table} to enclose the table's contents and
the table caption.  The contents of the table itself must go
in the \textbf{tabular} environment, to
be aligned properly in rows and columns, with the desired
horizontal and vertical rules.  Again, detailed instructions
on \textbf{tabular} material
is found in the \textit{\LaTeX\ User's Guide}.



\subsubsection{Application Programmer's Language (APL)}
Immediately following this sentence is the point at which
Table 1 is included in the input file; compare the
placement of the table here with the table in the printed
dvi output of this document.

To set a wider table, which takes up the whole width of
the page's live area, use the environment
\textbf{table*} to enclose the table's contents and
the table caption.  As with a single-column table, this wide
table will ``float" to a location deemed more desirable.
Immediately following this sentence is the point at which
Table 2 is included in the input file; again, it is
instructive to compare the placement of the
table here with the table in the printed dvi
output of this document.

\subsubsection{System Programmer's Language (SPL)}
Immediately following this sentence is the point at which
Table 1 is included in the input file; compare the
placement of the table here with the table in the printed
dvi output of this document.

To set a wider table, which takes up the whole width of
the page's live area, use the environment
\textbf{table*} to enclose the table's contents and
the table caption.  As with a single-column table, this wide
table will ``float" to a location deemed more desirable.
Immediately following this sentence is the point at which
Table 2 is included in the input file; again, it is
instructive to compare the placement of the
table here with the table in the printed dvi
output of this document.

\subsubsection{XSM Debugger}
Immediately following this sentence is the point at which
Table 1 is included in the input file; compare the
placement of the table here with the table in the printed
dvi output of this document.

To set a wider table, which takes up the whole width of
the page's live area, use the environment
\textbf{table*} to enclose the table's contents and
the table caption.  As with a single-column table, this wide
table will ``float" to a location deemed more desirable.
Immediately following this sentence is the point at which
Table 2 is included in the input file; again, it is
instructive to compare the placement of the
table here with the table in the printed dvi
output of this document.

\subsubsection{XFS Interface}
Immediately following this sentence is the point at which
Table 1 is included in the input file; compare the
placement of the table here with the table in the printed
dvi output of this document.

To set a wider table, which takes up the whole width of
the page's live area, use the environment
\textbf{table*} to enclose the table's contents and
the table caption.  As with a single-column table, this wide
table will ``float" to a location deemed more desirable.
Immediately following this sentence is the point at which
Table 2 is included in the input file; again, it is
instructive to compare the placement of the
table here with the table in the printed dvi
output of this document.

\begin{table*}
\centering
\caption{Some Typical Commands}
\begin{tabular}{|c|c|l|} \hline
Command&A Number&Comments\\ \hline
\texttt{{\char'134}alignauthor} & 100& Author alignment\\ \hline
\texttt{{\char'134}numberofauthors}& 200& Author enumeration\\ \hline
\texttt{{\char'134}table}& 300 & For tables\\ \hline
\texttt{{\char'134}table*}& 400& For wider tables\\ \hline\end{tabular}
\end{table*}
% end the environment with {table*}, NOTE not {table}!

As was the case with tables, you may want a figure
that spans two columns.  To do this, and still to
ensure proper ``floating'' placement of tables, use the environment
\textbf{figure*} to enclose the figure and its caption.
and don't forget to end the environment with
{figure*}, not {figure}!



\begin{figure}
\centering
\caption{A sample black and white graphic (.ps format) that has
been resized with the \texttt{psfig} command.}
\vskip -6pt
\end{figure}


\section{Roadmap}
This paragraph will end the body of this sample document.
Remember that you might still have Acknowledgments or
Appendices; brief samples of these

\section{Conclusions}
This paragraph will end the body of this sample document.
Remember that you might still have Acknowledgments or
Appendices; brief samples of these
follow.  There is still the Bibliography to deal with; and
we will make a disclaimer about that here: with the exception
of the reference to the \LaTeX\ book, the citations in
this paper are to articles which have nothing to
do with the present subject and are used as
examples only.
%\end{document}  % This is where a 'short' article might terminate

%ACKNOWLEDGMENTS are optional
\section{Acknowledgments}
This section is optional; it is a location for you
to acknowledge grants, funding, editing assistance and
what have you.  In the present case, for example, the
authors would like to thank Gerald Murray of ACM for
his help in codifying this \textit{Author's Guide}
and the \textbf{.cls} and \textbf{.tex} files that it describes.

%
% The following two commands are all you need in the
% initial runs of your .tex file to
% produce the bibliography for the citations in your paper.
\bibliographystyle{abbrv}
\bibliography{xos}  % sigproc.bib is the name of the Bibliography in this case
% You must have a proper ".bib" file
%  and remember to run:
% latex bibtex latex latex
% to resolve all references
%
% ACM needs 'a single self-contained file'!
%

\end{document}
